
\documentclass[a4paper]{article}

\usepackage[russian]{babel}
\usepackage[utf8]{inputenc}
\usepackage{cmap}
\usepackage{amsmath}
\usepackage{amssymb}
\usepackage{xspace}
\usepackage{graphicx}
\usepackage{sectsty}
\usepackage{fancyhdr}
\usepackage{longtable}
\usepackage{url}
\usepackage{appendix}
\usepackage{enumerate}

\newcommand{\EN}[1]{{#1}}
\newcommand{\CODE}[1]{{\ttfamily #1}}

\fancyhf{}
\fancyfoot[C]{\Large \thepage}
\renewcommand{\headrulewidth}{0pt}
\renewcommand{\footrulewidth}{0pt}

%%Modifying captions for figures and tables:
\makeatletter
\long\def\@makecaption#1#2{%
\vspace{\abovecaptionskip}%
\sbox{\@tempboxa}{\Large #1.~#2}
\ifdim \wd\@tempboxa >\hsize
{\Large #1.~#2}\par
\else
\global\@minipagefalse
\hbox to \hsize {\hfill {\Large #1.~#2}\hfill}%
\fi
\vspace{\belowcaptionskip}}

\renewcommand{\baselinestretch}{1.2}
%%\sectionfont{\LARGE}
\sectionfont{\Large}
\subsectionfont{\Large}
\subsubsectionfont{\Large}
\setlength{\belowcaptionskip}{6pt}
\makeatletter \renewcommand{\@biblabel}[1]{#1.\hfill}

\makeatletter 
\def\redeflsection{\def\l@section{\@dottedtocline{1}{1.5em}{7.8em}}} 
\renewcommand\appendix{\par 
\setcounter{section}{0}% 
\setcounter{subsection}{0}% 
\def\@chapapp{\appendixname}% 
\addtocontents{toc}{\protect\redeflsection} 
\def\thesection{\appendixname\hspace{0.2cm}\@Asbuk\c@section}} 
\makeatother 

\textwidth = 17cm
\oddsidemargin= 0 pt
\topmargin = -1cm
\headheight = 0cm
\headsep = 0cm
\textheight = 26.5cm

\begin{document}
\LARGE
\newpage

Здравствуйте!
Тема моей ВКР - “Реализация системы генерации статистических отчетов для IPTV провайдера”

\newpage
IPTV-провайдер - это организация, предоставляющая доступ к media-контенту абонентам в сетях передачи данных по протоколу IP.

Существует три основных бизнес-процесса, выполняемые провайдером:

Во-первых это Заключение договоров с правообладателями контента, такими как телевизионные холдинги, или кинокомпании. Суть этих договоров, заключается в том, что правообладатель передает права на предоставление своего контента абонентам провайдера на определенных условиях.

Второй БП -  Предоставление абонентам услуги IPTV, то есть непосредственно доступа к телевизионным каналам в режиме реального времени (в терминах провайдера это называется “Live”)

Третий - Предоставление доступа к статичному медиа-контенту, такому как фильмы или клипы, по запросу абонента (в терминах провайдера это называется “VOD” (Video-on-Demand)).
(55с)


\newpage

При принятии важных управленческих решений или для изменения маркетологической политики крупным IPTV-провайдерам необходимо анализировать и учитывать разного рода статистические показатели, отражающие состояние выполняемых бизнес-процессов в разные периоды времени.

Для удобного доступа к этим показателям необходима система, которая бы аккумулировала данные, требующиеся для построения этих показателей, и генерировала отчеты в соответствии с пожеланиями пользователя.

В ходе анализа требований к системе были выделены следующие крупные задачи:

--- Импорт данных
--- Реализация удобного интерфейса для поиска, просмотра и редактирования данных.

---  Расчет отчислений по договорам
---- Генерация отчетов по LIVE и VOD

---  Разграничение прав доступа к различным частям системы

\newpage

Система реализована в виде веб-приложения, что значительно упростило возможность масштабирования, обновления версий, и разграничения прав доступа.
В качестве основной среды разработки был выбран фреймворк Yii (PHP),\\
а в качестве хранилища данных --- СУБД Microsoft SQL Server.\\

\newpage

При анализе части предметной области, связанной с Live, то есть телевещанием, была спроектирована модель предметной области, изображенная на слайде в виде
ER-диаграммы в нотации Баркера

Особую роль здесь играют сущности:\\

ДоговорТВ --- договора с правообладателями (основными их характеристиками является период их действия и номер), договор заключается на определенном множестве регионов, и состоит из пакетов телеканалов.
Каждый пакет телеканалов содержит в себе градации
Градация содержит данные для проверки применимости при расчете отчислений а так же значения, определяющие размер отчисления по данной градации.

Так же на схеме представлена сущность ОтчисленияПоГрадации - отчисления в определенный месяц года, а так же разбивка отчислений по конкретным телеканалам.

\newpage

Кроме того для построения некоторых показателей используются данные ВыручкаТВ - размер выручки по пакету телеканалов в регионе в определенную дату, ПодписчикиПоПакетам - количество подписчиков в регионе в определенную дату и ДоступностьКанала - сущность, где хранится информация о том, какие каналы в каких пакетах были доступны.
На основе двух последних вводится избыточная сущность ПодписчикиПоКаналам, в которой хранится информация об усредненном количестве подписчиков определенного телеканала.

Большая часть данных для описанных выше сущностей импортируется из различных источников.
Исключением являются отчисления, которые расчитываются на основе данных договоров по запросу пользователя.


\newpage

При рассмотрении модели предметной области для VOD следует обратить внимание на следующие сущности:

--- Ассет - сущность, представляющая собой описание отдельного клипа, который может заказать абонент
--- ДоговорVOD - содержить описание договоров по VOD, и соответственно условий, по которым происходит расчет отчислений правообладателям --- ОтчисленияVod

Отчисления так же считаются на основе условий контракта, расположенных в УсловияДоговораVod.

\newpage

Большая часть статистических данных для отчета по VOD берется из сущности

--- ЗаказVOD - отдельные заказы ассетов пользователями ( основные характеристики - дата, регион, ссылка на ассет, выручка )

\newpage
На этом слайде можно увидеть форму параметров отчета

Основные:
--- Период отчета
--- Разбивка периода (для каждой части выбранного периода будут посчитаны показатели)

\newpage
--- Структура регионов, для которой будет строиться отчет: здесь можно выбрать вид дерева регионов, и сами вершины
Например, если выбрать пункт “Регионы, Филиалы, города”, каждый показатель будет посчитан отдельно для каждого уровня и выведен в виде дерева


\newpage
--- Также можно выбрать множество показателей, которые необходимо вычислить при построении отчета
Как видно на слайде, в случае с отчетом по VOD необходимо реализовать подсчет 33 различных показателей
--- Кроме того этой форме можно выбрать валюту, в которой нужно выводить значения показателей, связанных с денежными суммами.
(5мин)

\newpage
В общем случае отчеты представляют собой древообразную структуру вида:

  Дерево регионов -> Показатели

  Дерево регионов -> Пакеты телеканалов -> Показатели

  Дерево регионов -> Пакеты телеканалов -> Телеканалы -> Показатели


\newpage

Для каждого уровня дерева регионов вся эта структура повторяется - так на прошлом слайде были выведены показатели по Волжскому региону, а на этом - по Нижегородскому филиалу, который по сути является частью Волжского региона.

Каждый показатель содержит численные значения для каждого из периодов (столбцы с E по I) и последний столбец, значение EOP (End of Percent) - на сколько в процентах увеличился каждый показатель
6:00


\newpage

На этом слайде можно увидеть упрощенную диаграмму классов, демонстрирующую структуру отчета в общем виде.

По сути отчет - это древообразная структура, где каждая вершина представляет определенный уровень детализации

Здесь есть базовый класс StatReportNode --- по сути любая вершина дерева отчета,
и две конкретные реализации StatReportRegionNode для вершин-регионов и StatReportIndicator, в котором хранятся значения показателей.

Объект “отчет” содержит внутри ссылки корневые вершины-регионы и массив периодов, для которых были посчитаны показатели.

\newpage

На этом слайде изображена диаграмма последовательности процесса “Построение отчета”

--- Пользователь, находясь на странице с формой отчета, нажимает “Сгенерировать отчет”.

--- Так как процесс построения занимает существенное время, и результат не может быть получен в рамках одного HTTP-запроса, 
отчет генерируется в отдельном фоновом процессе, а пользователя перенаправляет на страницу ожидания отчета, которая раз в 10 секунд с помощью ajax-запросов, устанавливает, готов ли отчет.
и так до момента его полной генерации.

--- объект класса StatReportBuilder получает список объектов-вычислителей, каждый из которых добавляет в отчет соответсвующий ему показатель

--- Затем полученный объект-отчет передается в метод renderToFile класса StatReportExcelWriter, который сохраняет его в файле формата Excel, после чего пользователь может его скачать

\newpage
Стоит обратить особое внимение на загрузку классов-вычислителей показателей. По сути она реализована в виде системы плагинов.

1. CalculatorManager перед началом работы вызывает статический метод buildDefaultCalculators у всех файлов-классов, располагающихся в определенной директории.
В этом методе создаются и возвращаются объекты-вычислители для данного типа показателей

2. Так как некоторые показатели считаются на основе других показателей, CalculatorManager вызывает у полученных ранее объектов метод getDependencies(), возвращающий список зависимостей конкретного объекта.

3. На основе этих списков строится орграф зависимостей и производится его топологическая сортировка

4. Таким образом возвращается список объектов-вычислителей, такой что, если вызывать их в указанном порядке, то в момент вычисления показателя “с зависимостями”, они уже будут удовлетворены.

\newpage

В результате для того, чтобы добавить новый вид показателей, разработчику необходимо:

1. реализовать один из этих абстрактных классов-вычислителей (выбрать нужно в зависимости от вида показателя)

--- В случае если значение показателя для региона-родителя всегда равно сумме значений детей, то следует выбрать SummedIndicatorCalculator и реализовать метод, вычисляющий значение показателя в листьях дерева

--- В случае, если вам просто нужно добавить показатель в каждую вершину существующего дерева регионов, необходимо реализовать TraverseRegionIndicatorCalculator

--- Во всех остальных случаях следует использовать IndicatorCalculator

После чего нужно поместить реализацию в директорию с другими классами-вычислителями
и реализовать статический метод buildDefaultCalculators, возвращающий конкретные объекты

\newpage

В ходе выполнения ВКР были разработаны:

Структура базы данных 

Интерфейсы для поиска и редактирования данных в системе (для более чем 20 типов сущностей)

Расчет отчислений по категории LIVE

Генерация отчетов по LIVE и VOD

Система установлена и введена в эксплуатацию

\newpage
В ходе дальнейших работ планируется:

Реализовать графическую интерпретацию отчетов

Сделать возможным просматривать отчет в течение времени его построения

Оптимизация процесса генерации отчетов за счет уменьшения количества избыточных данных



\end{document}
