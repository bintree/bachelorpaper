\section{Постановка задачи}

В этой главе подробно описаны основные требования к системе, разрабатываемой в~рамках данной работы,
а так~же в~контексте конкретных задач рассмотрены основные понятия предметной области. 

В последующих разделах этой главы детализированы следующие основные задачи:

\begin{enumerate}
\item{
  Реализация модуля импорта данных.
}
\item{
  Реализация подсистемы, обеспечивающей удобный интерфейс для~поиска, просмотра и модификации импортированных данных.
}
\item{
  Реализация алгоритма расчета отчислений правообладателям. 
}
\item{
  Генерация статистических отчетов в соответствии с~пожеланиями пользователя.
}
\item{
  Реализация механизмов авторизации, позволяющих разграничивать права доступа пользователей к~различным частям системы.
}
\end{enumerate}

Кроме того со стороны заказчика были выдвинуты следующие нефункциональные требования к~системе:

\begin{enumerate}
\item{
  Система должна быть реализована в~виде веб-приложения.
}
\item{
  В~качестве хранилища данных должна быть использована СУБД Microsoft SQL Server.
}
\item{
  Должен быть продуман механизм первоначальной установки системы и обновления версий 
  с~учетом ограничения доступа со стороны исполнителей к~веб-серверу и серверу~БД.
}
\end{enumerate}

Для дальнейшего ознакомления с работой необходимо ознакомиться с некоторыми базовыми терминами предметной области.

\textbf{Live} \textit{(синонимы: TV, телевещание)} --- часть деятельности провайдера, связанная с вещанием
телеканалов в режиме реального времени.

\textbf{VOD} \textit{(синонимы: Video-on-Demand, статика)} --- часть деятельности провайдера, связанная 
с предоставлением доступа к статичному видео-контенту.

В системе \textbf{Live} и \textbf{VOD} должны быть представлены в виде двух различных разделов приложения,
так как процессы, протекающие в контексте этих частей, практически не пересекаются.

\label{gloss:asset}
\textbf{Ассет} \textit{(синонимы: Asset, клип)} --- единица статичного контента, множество которых составляет
VOD-библиотеку провайдера.

\textbf{Правообладатель} \textit{(синонимы: Лицензиар)} --- организация, предоставляющая контент провайдеру

\textbf{Биллинг провайдера} --- система, в которой происходит учет информации об использовании абонентами услуг провайдера, их тарификация, 
выставление счетов абонентам, обработка платежей. Для проектируемой в рамках данной работы системы биллинг является основным источником
данных, используемых при построении отчетов.

\textbf{База абонентов} \textit{(синонимы: база, base)} --- в разных случаях так называют число или множество абонентов,
удовлетворяющих определенному критерию. Например число абонентов определенного пакета телеканалов.

\subsection{Импорт данных}
Бизнес-процессы, исполняемые IPTV-провайдеров, порождают большое количество разнородных данных, 
на~основе которых должны строиться статистические отчеты.

В соответствии с числом базовых бизнес-процессов, описанных во Введении, существует два
основных вида импортируемых данных:

\begin{enumerate}
\item{
Структура договоров с правообладателями, которая детально описывает размер 
отчислений правообладателям при разных параметрах.
}
\item{
Данные, связанные с предоставлениям услуг абонентам: размер выручки провайдера, количество подписчиков и заказов VOD. 
}
\end{enumerate}

В этом разделе детально описаны источники данных, структура этих данных и понятия предметной области,
связанные с этими данными.

\subsubsection{Справочники}
Справочники --- это сущности, непосредственно связанные с деятельностью провайдера, однако не играющие ключевой роли
в~оценке состояния бизнес-процессов. Предполагается, что данные из~справочников не подвержены регулярным изменениям,
и их наполнение нужно произвести лишь однажды при инициализации системы.

Ниже описаны основные типы справочников:
\paragraph{Организации}
Список юридических лиц с их основными реквизитами. 

\paragraph{Дерево Территорий}
\label{par:regions}
Большая часть сущностей, используемых при построении отчетов, имеет связь с~конкретными территориями,
в которых провайдер осуществляет свою деятельность. Множество территорий образует дерево 
с четырьмя уровнями, причём в каждый уровень включаются объекты, непосредственно подчинённые объектам предыдущего уровня:

\begin{itemize}
\item{
  \textit{Страна} --- на данный момент деятельность провайдера ведется только в РФ, но при проектировании
  стоит учитывать возможность выхода на рынки других стран.
}
\item{
  \textit{Регион} --- страны разделяются на множество регионов. В случае РФ регионы соотвествуют
  крупным единицам административно-территориального деления, таким как Сибирский Федеральный округ 
  или Южный федеральный округ.
}
\item{
  \textit{Филиал} --- территориальная единица областного или краевого масштаба, характеризуется наличием
  местного филиала провайдера.
}
\item{
  \textit{Город} --- лист дерева регионов, соответствует населенным пунктам, в которых ведется деятельность провайдера.
}
\end{itemize}

Далее в работе термин ``Регион'' будет использоваться, как синоним понятия ``Территория''.

\paragraph{Телеканалы}
Список телеканалов, вещание которых производится провайдером. Каждый из каналов характеризуется его именем
и уникальным идентификатором в биллинге провайдера.

\paragraph{Пакеты телеканалов}
Каждый абонент IPTV-провайдера при подключении выбирает один из пакетов телеканалов, на~который
он подписывается. В контексте справочников пакеты характеризуются их именем и уникальным идентификатором
в~биллинге провайдера.

\subsubsection{Договоры с правообладателями}

Договоры с правообладателями --- это сущности, описывающие условия предоставления контента провайдеру.
В целом структуры договоров по Live и VOD похожи, но каждый из этих видов обладает своей спецификой.

\paragraph{Договоры по Live}
Основные характеристики договоров этого типа:

\begin{itemize}
\item{
  \textit{Правообладатель} - организация, с которой заключен договор.
}
\item{
  \textit{Юр. лицо} -  организация со стороны провайдера, на имя которой заключен договор.
}
\item{
  \textit{Номер договора} ---  уникальный идентификатор со сквозной нумерацией.
}
\item{
  \textit{Даты} --- дата заключения договора и его длительность.
}
\item{
  \textit{Территории} --- множество вершин дерева территорий, для которого актуален договор.
}
\end{itemize}

Каждый из договоров по Live содержит в себе \textit{пакеты телеканалов} (один и более), права на вещание которых обеспечивает договор.

Для каждого из пакетов должны быть описаны градации отчислений (не менее одной в пакете), устанавливающие условия и размер
отчислений правообладателю. Структура градаций подробнее рассмотрена в разделе \ref{live:deducts}.

На момент проектирования системы договоры по Live у провайдера хранились в виде файла формата Microsoft Excel 2007, в~связи с~чем
необходимо было реализовать модуль для регулярного импорта договоров со всеми связанными с ними сущностями из этого файла.

\paragraph{Договоры по VOD} в целом изоморфны договорам первого типа, за исключением того, что условия договоров распространяются на множества ``ассетов''
вместо пакетов телеканалов.

\subsubsection{Статистические данные по Live}
Существует несколько видов статистических данных, связанных с Live, которые система должна регулярно импортировать
с помощью веб-сервисов биллинга.

\textbf{Количество подписчиков пакетов} --- множество кортежей, каждый из которых содержит информацию о количестве подписчиков
определенного пакета телеканалов для конкретной даты в одном из листьев дерева регионов (см. \ref{par:regions}).

\textbf{Доступность телеканала} --- множество кортежей, каждый из которых содержит информацию о доступности телеканала в
определенном пакете для конкретной даты в одном из листьев дерева регионов.

\label{stat:subscribers}
Информация о количестве подписчиков определенного телеканала в пакете не может быть получена из биллинга по той причине, что этой информации там нет, 
так как при подключении абоненты выбирают именно пакеты каналов. Но при этом можно вычислить усредненное значение этой величины 
в~рамках определенной даты, территории, пакета и телеканала по формуле:
$$Subscribers_{channel} = \frac{Subsribers_{packet}} {Channels_{packet}}, $$
где $Subscribers_{channel}$ --- искомая величина,  $Subsribers_{packet}$ --- количество подписчиков пакета, 
а $Channels_{packet}$ --- количество доступных в пакете каналов.

Еще одной важной импортируемой из биллинга величиной является \textbf{Выручка} --- множество кортежей, каждый из которых содержит информацию о
суммарной абонетской плате, полученной по определенному пакету телеканалов для конкретной даты в одном из листьев дерева регионов.
 
Веб-сервисы биллинга предоставляют вышеописанные данные в формате XML со строго заданной схемой. Система должна ежедневно импортировать эти данные,
сохраняя записи за последние пять лет.

\subsubsection{Статистические данные по VOD}
Множество доступных абонентам \textit{Ассетов}(см. \ref{gloss:asset}) должно еженедельно импортироваться в систему из 
биллинга провайдера, для этого также предоставляется веб-сервис, возвращающий данные в формате XML.

Основным видом статистических данных для VOD являются \textit{Заказы абонентов}, каждый из которых связан с одним из ассетов,
а также содержит информацию о дате, когда был сделан заказ, регионе абонента, и стоимости, которая списана со счета абонента.

Импорт заказов также должен производится из биллинга ежедневно.

\subsection{CRUD Интерфейс} 
Для каждого типа сущностей, введенного в рамках системы необходимо разработать удобный интерфейс для добавления, 
просмотра, редактирования, удаления и поиска записей.

\subsubsection{Список объектов}
\label{crud:list}
Должен быть реализован постраничный вывод каждого из множеств сущностей с возможностью фильтрации и сортировки
по значениям полей, кроме того необходим механизм удаления одновременно нескольких объектов из списка.
Для каждой сущности на странице должны быть выведены основные поля, характеризующие объект.
Если характеристикой объекта является связь с другими сущностями в формате ``многие к одному'',
нужно вывести наименование связанного объекта, а в форме фильтрации обеспечить возможность выбора объекта
для фильтрации. 

В случае хронологического характера данных по умолчанию сортировка должна производиться по дате в убывающем порядке.

\subsubsection{Формы создания и редактирования}
Создание и редактирование объектов должно происходить на отдельной странице с выведенной на ней формой для ввода данных.
Следует уделить особое внимание верификации и контролю целостности данных при сохранении форм, а в случае неверно
введенных данных уведомлять пользователя об ошибке, выделяя те поля формы, где произошла ошибка.

Существуют следующие основны типы ``проверки корректности данных'':
\begin{itemize}
\item{
  \textit{Обязательное значение} --- поле обязательно должно быть заполнено. Например номер договора. 
}
\item{
  \textit{Формат} --- значение поля должно удовлетворять определенному формату (число, дата и т.п.).
}
\item{
  \textit{Уникальность} --- значение поля или совокупность значений полей должна быть уникальна в рамках
данного типа сущности.
}
\item{
  \textit{Частные случаи корректности данных} --- для некоторых сущностей существуют особые правила заполнения полей,
  когда то или иное значение считается корректным в зависимости от значений других полей. 
  Например в градациях договоров по Live не могут быть одновременно заполнены поля ``Фиксированный платеж`` 
  и ``Платеж за абонента'', но при этом одно из полей должно быть непустым.
}
\end{itemize}

Для договоров с правообладателями необходимо выводить связанные с ними условия(градации и пакеты)
в виде табличной detail-формы внутри которой также должна быть возможность фильтрации и сортировки по выбранным полям.

\subsection{Алгоритм расчета отчислений правообладателям}
\label{live:deducts}
\textbf{Отчисления правообладателям} --- это сумма, которую провайдер ежемесячно выплачивает лицензиарам
в соответствии с условиями договоров. В рамках проектируемой системы необходимо реализовать алгоритм расчета
этой величины, а также создать интерфейс для вычисления отчислений за произвольный месяц на основе текущей 
структуры договоров. Полученные значения должны сохраняться в базе данных и выводиться пользователю 
в виде списка объектов (см \ref{crud:list}).

Так как для первой версии системы предусмотрена автоматическая генерация отчислений 
только для Live, далее в этом разделе будет описан алгоритм только для этого типа договоров.

Сумма для для конкретного правообладателя расчитывается следующим образом:

$$Deducts_{licensor} = \sum_{contract \in Contracts} Deducts(contract)$$,

то есть как сумма отчислений по активным договорам этого лицензиара. 

Отчисления по отдельному договору вычисляется как сумма отчислений по каждой из его применимых по сроку и базе градаций.

Для определения применимости градации по сроку задана характеристика ``Срок действия градации'', состоящая из двух дат.
Если этот параметр заполнен, то градация применима только в том случае, если период, 
для которого расчитываются отчисления пересекается со сроком действия градации. 

Применимость по базе определяется в зависимости от вида расчета, но общая суть заключается в том, 
что количество абонентов градации должно лежать в интервале характеристик градации ``Мин. число абонентов'' и
``Макс. число абонентов'', в случае если они заданы.  

Характеристика градации ``Вид расчета'' определяет каким образом будет вычислена сумма для каждой градации,
каждый из видов описан далее. По сути разница состоит в том, будет ли сумма считаться отдельно
для каждого телеканала пакета, или для всей совокупности одновременно.

\paragraph{Раздельный вид расчета}
В случае раздельного вида расчета для каждого телеканала из пакета, которому принадлежит градация,
расчет производится отдельно, затем полученные величины суммируются:

$$Deducts_{gradation} = \sum_{channel \in Channels_{packet}} deductbase(gradation, Subscribers_{channel})$$

Методы вычисления величины $Subscribers_{channel}$ (база канала) описаны в \ref{stat:subscribers}.
Вид функции $deductbase$(отчисления по базе абонентов) будет описана ниже. 

Градация считается применимой по базе в случае, если хотя бы база одного из каналов принадлежит отрезку 
[`Мин. число абонентов'', ``Макс. число абонентов''].

В случае суммарного вида расчета существует возможность получить корректную сумму отчислений по каждому из 
телеканалов.

\paragraph{Суммарный вид расчета}
Для суммарного вида расчета вычисляется величина:
$$Subscribers_{gradation} = \max\limits_{channel \in Channels_{packet}} Subscribers_{channel}$$

Сумма отчислений по градации устанавливается как:
$$Deducts_{gradation} = deductbase(gradation, Subscribers_{gradation})$$

Градация считается применимой по базе в случае, если величина $Subscribers_{gradation}$ принадлежит отрезку 
[`Мин. число абонентов'', ``Макс. число абонентов''].

Для этого вида расчета нельзя точно установить точную сумму отчислений по каждому каналу в отдельности,
однако в рамках разрабатываемой системы для использования в отчетах можно вычислять эту величину как частное 
суммы отчислений и числа каналов в пакете.

\subsubsection{Схемы расчета отчислений}
Сумма отчисления по градации и базе абонентов ($deductbase$) расчитывается в зависимости от 
``Cхемы расчета'' --- еще одной характеристики градации. Существует две схемы расчета ``Платеж за абонента''
и ``Фиксированный платеж''. При валюте договора отличной от рублей необходимо перевести все денежные величины
в рублевый эквивалент по актуальному на период расчета курсу (последний день месяца).

В случае схемы ``Фиксированный платеж'' отчисления по базе не зависят от этой величины, и равны величине ``Фикс.платеж'' т.е. 
$ deductbase(gradation, subscribers) = Fix_{gradation} $.

Расчет по схеме ``Платеж за абонента'' должен происходить по следующей формуле:
$$ deductbase(G, C) =  \max(C, GR_{G}) \times AP_{G} \times \lfloor \frac{CalcDays}{PeriodDays} \rfloor \times \frac{DeductPercent_{G}}{100}$$

В этой формуле $G$ --- градация, по которой устанавливается сумма, $C$ - количество подписчиков за расчетный период,
$GR_{G}$ --- характеристика градации ``Минимальное гарантированное число абонентов'',
$AP_{G}$ --- характеристика градации ``Платеж за абонента'',
$CalcDays$ --- число дней расчетного периода, в которые данная градация была актуальна,
$PeriodDays$ --- число дней в расчетном периоде,
$DeductPercent_{G}$ --- характеристика градации ``Процент отчислений''.

