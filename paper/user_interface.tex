\subsection{Пользовательский интерфейс}

HTML-разметка приложения разработана с помощью ``Twitter Bootstrap'' -- набора инструментов
предоставляющего набор CSS и HTML шаблонов, позволяющих быстро создавать пользовательские
интерфейсы для веб-приложений.

%%TODO: скрин меню

Главное меню приложения, располагающееся вверху экрана, содержит три основных пункта:
\begin{itemize}
\item{
\textbf{Live}
}
\item{\textbf{VOD}}
\item{\textbf{Системные настройки}}
\end{itemize}.

Первые два содержат подпункты со ссылками на CRUD-интерфейсы для редактирования
соответствующих сущностей из одноименных разделов предметной области.

Раздел ``Системные настройки'' содержит в качестве подпунктов ссылки на интерфейс для 
редактирования пользователей и справочники, общие для VOD и Live.

Меню приложения реализовано согласно типовому решению ``Двухэтапное представление''\cite{fowler},
перед генерацией HTML-кода по шаблону в контроллере происходит инициализация объекта 
\textit{CMenu} --- компонента Yii, в котором описывается древообразная структура меню,
после чего полученный объект передается в представление для генерации разметки. 

\subsubsection{CRUD-интерфейс}
Всего в системе представлено около двадцати типов сущностей, для которых было необходимо
предоставить CRUD-интерфейс в соответствии с требованиями из \ref{section:crud}.

В ходе разработки такого рода разделов был выработан следующий набор рекомендаций
разработчику при создании нового интерфейса:

%%todo: скрин формы договоров

\begin{enumerate}
\item{
  С помощью веб-интерфейса Giix(компонент Yii для генерации кода) на основе уже существующей таблицы 
в БД генерируются классы модели, описывающие параметры отображения для ORM (см. \ref{section:yii}).
}
\item{
  В полученных классах уже будут описаны базовые правила проверки корректности данных модели,
основанные в частности на типах атрибутов таблицы. Но в большинстве случаев необходимо добавлять
дополнительные проверки, для этого необходимо перекрыть метод \textit{rules()} модели, в котором
могут быть описаны дополнительные правила валидации.
}
\item{
  Для установки названий полей на русском языке необходимо 
соответствующим образом перекрыть метод \textit{attributeLabels()} модели, после чего добавить
русскоязычные названия в файл переводов. 
}
\item{
 С помощью веб-интерфейса Giix на основе класса модели генерируется код контроллеров, форм
и представлений, реализующих простой CRUD-интерфейс.
Полученный код уже готов к использованию и содержит интерфейс просмотра списков записей 
с возможностью фильтрации и сортировки по значениям полей, а также формы для создания,
редактирования и просмотра сущностей. В случае наличия связи ``многие-к-одному'' у модели 
вместо поля с идентификатором связанной сущности в форме отображается lookup-элемент.
Для задания значения дат в сгенерированных формах используется специальный виджет,
реализованный на основе jQuery Datepicker.
}
\end{enumerate}

Полученный набор форм и представлений чаще всего нуждается в доработке: 
\begin{itemize}
\item{
Не все поля модели должны быть доступны пользователю, некоторые из них носят служебный характер,
и после генерации их нужно убирать из форм и таблиц поиска.
}
\item{
В некоторых случаях порядок расположения полей в сгенерированной форме не удобен
пользователю.
}
\item{
Некоторые множества сущностей, такие как например договоры с их градациями должны быть представлены
в виде master-detail форм.  
}
\end{itemize}

Одним из существенных недостатков кодогенератора Giix является отсутствие возможности повторной генерации
без потери изменений, внесенных в процессе доработки.
