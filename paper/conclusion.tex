\section*{Заключение}
\addcontentsline{toc}{section}{\hspace{7mm}Заключение}

По~результатам выполненной работы сделаны следующие выводы:

\begin{enumerate}
\item {
Автором проведено исследование формата базы данных Wikipedia,
и реализован программный пакет, позволяющий получить доступ к~тексту
и отдельным элементам разметки отдельных страниц.
}
\item {
Реализован модуль для~сохранения статей из~архива Wikipedia
в СУБД~MySQL с~возможностью последующего доступа к~произвольным страницам
по~их~заголовку или идентификатору. 
}
\item {
Проведены исследование и условная классификация существующих 
алгоритмов определения меры семантической близости предложений. 
Проанализированы и показаны достоинства и недостатки исследованных методов.
}
\item {
Реализован один из рассмотренных алгоритмов и запущен для~набора 
пар предложений, близость которых была оценена людьми. 
Среднее значение модуля разности оценок людей и алгоритма составило 0,15.
Создано консольное приложение, выполняющее поиск предложений 
в~определенной статье Wikipedia, выбирая те~из~них,
которые наиболее близки по~смыслу к~предложению, введенному пользователем.
}
\item {
В~процессе реализации были изучены программные продукты для работы
с~естественными языками, такие как CoreNLP и WordNet.
} 
\end{enumerate}

В~ходе дальнейших исследований предполагается изучение более производительных БД, таких~как 
Lucene и MongoDB, которые лучше приспособлены для~поиска в~массивах текстов подобного размера.

Выдвинуты предложения по улучшению реализованного алгоритма:

\begin{enumerate}
\item{
При~оценке степени близости отдельных вершин дерева связей предложения учитывать 
их расстояние от~корня.
}
\item{
Использовать более эффективные алгоритмы определения изоморфности деревьев связей.
}
\item{
Модифицировать алгоритм, приспособив его для поиска предложения в~тексте.
В~этом случае искомая величина будет мерой вхождения одного предложения в~другое,
и в~соответствии с этим производить необходимые вычисления.
}
\end{enumerate}
