\section*{Введение}
\addcontentsline{toc}{section}{\hspace{7mm}Введение}

``IPTV'' --- это технология, позволяющая передавать телевизионное изображение и звук по компьютерным сетям с использованием протокола IP.

За последние несколько лет многие интернет-провайдеры регионального и федерального уровней существенно 
расширили спектр своих услуг, включив в него предоставление абонентам доступа к цифровому телевидению.
По большей части столь массовое освоение рынка связано с тем, что вся инфраструктура, то есть сети передачи данных, 
уже существовали, и единственное, что оставалось сделать --- это распространить среди своих абонентов специальные приставки, 
с помощью которых они могут получать доступ к IPTV-сетям.

Таким образом ``IPTV-провайдер'' --- это организация, предоставляющая абонентам услугу цифрового телевидения с использованием IPTV. Как правило они предоставляют три вида услуг:
\begin{itemize}

\item {
Телевещание в режиме реального времени (с использованием IP Multicast).
}

\item{
``Time Shifted TV'' --- услуга предоставления доступа к записанным телевизионным программам.
}

\item{
``VoD'' (\textit{Video on Demand}) --- услуга предоставления доступа к статичному видео-контенту, необязательно связанному с телевещанием.
}

\end{itemize}

Cуществует два базовых бизнес-процесса, выполняемые провайдером:

\begin{enumerate}

\item{
Заключение договоров с правообладателями контента, такими как телевизионные холдинги, или кинокомпании. 
Суть этих договоров, заключается в том, что правообладатель передает права на предоставление своего контента
абонентам провайдера на определенных условиях, чаще всего с периодическим взиманием отчислений в пользу правообладателя.
}
\item{
Предоставление платных услуг, описанных выше, конечным абонентам.
}

\end{enumerate}

\vspace{0.5cm}

При принятии важных управленческих решений или для изменения маркетологической политики крупным IPTV-провайдерам 
необходимо анализировать и учитывать разного рода статистические показатели, отражающие состояние этих бизнес-процессов 
в разные периоды времени.

Для удобного доступа к этим показателям необходима система, которая бы аккумулировала данные,
требующиеся для построения этих показателей, и генерировала отчеты в соответствии с пожеланиями пользователя.

Целью данной работы является реализация такой системы, предназначенная для использования одним из IPTV-провайдеров
федерального уровня.

Квалификационная работа состоит из двух глав.

В первой главе рассмотрены основные понятия предметной области, а также детализированы цели и задачи данной работы. 

Во второй рассмотрены уже существующие решения и приведены детали реализации полученной в рамках данной работы системы.

В заключении описаны результаты работы и сформулированы перспективы дальнейшего развития системы.

